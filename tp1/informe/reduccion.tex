\section{Reducción de dimensionabilidad}
Experimentamos con la técnica de selección univariada conocida como \textit{Ranking de atributos} para eliminar las palabras que no aportan información relevante
para la clasificación. Esta técnica consiste en generar un ranking de atributos evaluando cada uno por separado para luego elegir los K mejores rankeados. \\
En un primer intento tratamos de realizar la reducción utilizando la combinación de dos tipos de selecciones posibles, Ranking de atributos y PCA.
 Pero esto no fue posible ya que nos encontramos con incompatibilidades en las funciones de las librerías.
 Es por esto que decidimos experimentar unicamente con el selector \textit{KBest}, quedandonos con los primeros 100 del ranking.

{\Large Ver si aumentamos la cantidad de palabras inicial o reducimos los 100}

{\Large Ver si agregar un histograma de los scores}

\begin{itemize}
\item \textbf{score\_func:} \textit{chi2, ANOVA F-value} - La métrica de scoring
\end{itemize}
