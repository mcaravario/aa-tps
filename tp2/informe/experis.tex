\section{Experimentos y Resultados}
Para la experimentación realizada en este trabajo decidimos comenzar implementando la estrategia \textbf{Grid search},
la cual nos permitio encontrar la mejor configuracion de hiperparametros entre un subconjunto creado por nosotros. Estos
hiperparametros son: \textit{learning rate} y \textit{discount} para el algoritmo de \textbf{QLearning}, y el epsilon para
la politica de control \textbf{$\epsilon-greedy$}. La busqueda fue realizada entre los siguientes valores de cada uno:

\begin{itemize}
  \item  \textbf{$\epsilon$} = 0.1, 0,2
  \item \textbf{learning\_rate} = 0.1, 0.2, 0.4, 0.6, 0.8, 0.9, 1.0
  \item \textbf{discount} = 0.5, 0.6, 0.7, 0.8, 0.9
\end{itemize}


Obteniendo como mejor resultado la siguiente configuracion:

\begin{itemize}
  \item  \textbf{$\epsilon$} = 0.1
  \item \textbf{learning\_rate} = 0.4
  \item \textbf{discount} = 0.9
\end{itemize}


\subsection{Experimento 4 en linea en tablero de $7\times6$}
En el siguiente experimento decidimos entrenar con el juego completo de 4 en linea, en su formato original de un tablero
de $7\times6$. Para este caso decidimos experimentar con dos politicas de control: \textit{$\epsilon-greedy$} y
\textit{softmax}. En ambos casos realizamos 500000 iteraciones de juego, y para cada caso entrenamos a nuestro agente
contra un jugador random y contra otro agente de Qlearning. \\
Los parametros utilizados tanto para \textit{$\epsilon-greedy$}, como para \textit{softmax} fueron los obtenidos por la
estrategia de Grid search previamente explicada.

Los resultados obtenidos en cada caso fueron:

\todo{Poner resultados Qlearn vs Qlearn(epsilon greedy)}

\todo{Poner resultados Qlearn vs Qlearn(softmax)}

\todo{Poner resultados Qlearn vs random(epsilon greedy)}

\todo{Poner resultados Qlearn vs random(softmax)}

\subsection{Experimento 3 en linea en tablero de $6\times5$}
